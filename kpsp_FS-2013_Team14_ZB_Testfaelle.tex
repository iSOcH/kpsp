\subsection{Key Generator}
Der Keygenerator soll zufällige RSA-Keys erzeugen. Wir können ihn nach der Kompilation mittles \emph{ghc -XOverloadedStrings hskeygenerator.hs} aufrufen, um Keys für unseren Sender und den Empfänger zu generieren:

\begin{lstlisting}
[iso@iso-t530arch tmp]$ ./hskeygenerator 
Enter filename:
sender
[iso@iso-t530arch tmp]$ ./hskeygenerator 
Enter filename:
receiver
\end{lstlisting}

Die Inhalte der erzeugten Schlüsselfiles sehen dann wie folgt aus:
\begin{lstlisting}
[iso@iso-t530arch tmp]$ cat senderRsaPubKey 
----BEGIN RSA PUBLIC KEY----
AQAB,uLEhNs/uRmG9
----END RSA PUBLIC KEY----

iso@iso-t530arch tmp]$ cat senderRsaPrivKey 
----BEGIN RSA PRIVATE KEY----
Ia9kwGVnj+Bh,uLEhNs/uRmG9
----END RSA PRIVATE KEY----

[iso@iso-t530arch tmp]$ cat receiverRsaPubKey 
----BEGIN RSA PUBLIC KEY----
AQAB,OSD0b/N9Btp5
----END RSA PUBLIC KEY----

[iso@iso-t530arch tmp]$ cat receiverRsaPrivKey 
----BEGIN RSA PRIVATE KEY----
EbzFqCfx729x,OSD0b/N9Btp5
----END RSA PRIVATE KEY----
\end{lstlisting}

Wie erwünscht enthalten die jeweils zueinander gehörigen Private- und Public-Keys den selben Wert für $N$ (\glqq{}uLEhNs/uRmG9\grqq{} bei sender, \glqq{}OSD0b/N9Btp5\grqq{} bei receiver). Bei beiden Schlüsselpaaren wird für $e$ der Wert 65537 verwendet, $d$ hingegen unterscheidet sich zwischen den Schlüsselpaaren.

In diesem Fall wurden die folgenden Schlüssel erzeugt. Die Werte können mittels folgendem Befehl (nach dem Laden von hsencrypt.hs) zurückgewonnen werden:
\begin{lstlisting}
Kpspcrypto.Serial.asInt $ Kpspcrypto.Base64.decode "Ia9kwGVnj+Bh"
\end{lstlisting}

\begin{itemize}
	\item Sender e: 65537, d: 621380992428485369953, n: 3406964452648185913789
	\item Receiver e: 65537, d: 327197112392121020273, n: 1053839058196540873337
\end{itemize}

\subsection{Ver- und Entschlüsselung von Nachrichten}
\subsubsection{Vorbereitung}
Mit den folgenden Befehlen werden (unter Linux) zufällige Testnachrichten erstellt und deren Hashwerte (für den späteren Vergleich) ermittelt:

\begin{lstlisting}
[iso@iso-t530arch tmp]$ dd bs=1k count=1 if=/dev/urandom of=1k.msg
1+0 records in
1+0 records out
1024 bytes (1.0 kB) copied, 0.000435418 s, 2.4 MB/s
[iso@iso-t530arch tmp]$ dd bs=1k count=100 if=/dev/urandom of=100k.msg
100+0 records in
100+0 records out
102400 bytes (102 kB) copied, 0.0180468 s, 5.7 MB/s
[iso@iso-t530arch tmp]$ sha256sum 1k.msg 100k.msg 
fd6df86538db0013a9f943b2d8a03d52a5d6a40cbe3243408167dc15e29a855d  1k.msg
1b13213582917f5b36751ab66bf22ae19233e259de319181acbeb64d712a3559  100k.msg
\end{lstlisting}

\subsubsection{Verschlüsselung ECB}
Für diese Tests werden die zuvor erzeugten Schlüssel verwendet. Wir verschlüsseln die beiden Nachrichten mit den öffentlichen Schlüssel des Empfängers und signieren den Inhalt der Nachricht mit Hilfe des privaten Schlüssels des Senders:

\begin{lstlisting}
[iso@iso-t530arch tmp]$ ./hsencrypt RSA SHA256 AES256 ECB senderRsaPrivKey receiverRsaPubKey 1k.msg 
[iso@iso-t530arch tmp]$ ./hsencrypt RSA SHA256 AES256 ECB senderRsaPrivKey receiverRsaPubKey 100k.msg
\end{lstlisting}

Den Inhalt des 1k-Files schauen wir uns an (Zeilenumbrüche innerhalb der Msg-Teile wurden manuell hinzugefügt):
\begin{lstlisting}
[iso@iso-t530arch tmp]$ cat 1k.msgEncrypted 
----BEGIN KEYCRYPTED RSA----
KB25vJ46fR06,ISPN+nHCYyt3,AdvzIdRlMOpz,LFpPe8rS7WCM,NOgghBG7w74E,
Li21sI2vAGBH,GfDWpqTKuuP+,KgFI/+tsr+NX
----END KEYCRYPTED----

----BEGIN MSGCRYPTED AES256 ECB----
NC5ja4U25ZMXVRQAFlYn0aIJcr26nx+gEYlARbw52QHDoQnywSEZRTFXuQ5K4kSvx1AiW
L+s719SBx0OGi/o+JOKp+cutp4ArUSHK7aWdBVrJpTi6a0Bj0fyKr5xsAVGZAfC3XCx14
li16+05/p+fOwN8LzixmqQ2R0T49n+iI7n/9Uw1wu1UbYPfjgBIeXT8HGHc+GevYAkrXv
QHXVdbFaBZXQBQWWcfr+A0rbfOkxG2bDh5FwR7WF+7PFDK7h2peXSFJ/nFu4SSLBVbEED
PFbbhG/fS+IcQ4y5Mu5/ICfc2WeZg8r83cRhu2nDJouOzQXM8qxvLrpY0IZ5xhbB2b0mT
vg01KaD9mp4UrtxDvsqLs9kwuGgMKruqKolM3C49zx3uhBSb0T4uF/2hgowPuhrN0Xppd
ytMuMvstvJGImPEuj+CAFJ6GbFbBQj5xwlvsgx3tsYiCzTe6A62m0yuATioFDAuGB7A6a
tdgDLiNyfV3oNFGuBukIe1UAZyz5xDWyfCbR0bG8Ok+38oXqMzRrdyv/zhtwZaugnhnLa
H/Eu8H3AmqoMz6hVp6xDtX72HcYu/FXOaRtFZsH6PWEPJdu02uCGQ7G/1dUM2frG6SPSj
lfRxPpUmQTkoDOtk51t/nv2BQUqcwYuDHWHzL4wIw/+wAY1xe5LU/WiPd9/3Galv4saan
wDsrpFqvbgMdPkOW6wNHBu4Vl8KG4TPRpymwYKvwhg2hIU01RT+zJ+ezVsgrl74cNj5DP
3NrlNdiKhRVmd7oAJPHde2g6ZhdB1YrcAhSsiVOq118UiKpbn/LfobGUKOTa51/wollRs
riC4uU4ULyXix0C+WHLHdGd/xwaGmoBSBf0h+I/fUZ97xw6fgdN0oyfek75tdQpiPI18X
NHVwYd2qAdH/BTuM+ODuqjgPcuunUzJXfGJpQVDBPPQh6akzIyyHfQMBJN0N4o1jfUKL6
CFWaHmXRQpCnVkFwsKP5NlOYfpjsY7N34OmrqAOZP/wIBYs+HjcF4YxirE98iOcII5Rsf
O5i/wtQiyXgc/kFkY93uluwrZQT0OJKWzctH0isSIRPqGk3ySliB7Ceh7spqMIVPuPpn4
W63yFtf9XrzDtW8gMv/roX/0HJmeoW1ysjPjJ+teT6lZ50REmh4/LFGjkRQx2n1x+wflC
wwTusYA/1I/lzZRDvHhpljHGX1x8H9fYJekBBKYmPu6ufS0uEFS2UwaHFHalwIXuE2kVj
3Bfiop5rqgZZimVfoFKnFjo+V0d4SXuuqAVv+IFnPorG4nXThHf2TN5h6cn40xrlOfrM3
iRLB/CMCuvyvaV0luRLSKmYbrjDQr9ShFhLf+ER6Mp0eVxp8GfxnKdkCCpiHjrTM4eVLM
IoIMH1s=
----END MSGCRYPTED----

----BEGIN SIGNATURE RSA SHA256----
i7XYWSqqv8b7,P3ERQK9xnLYD,kMwA8TdQpUcb,OmazloPVsEod,YllxmJXn8mmf,
FmK2J2p4xgnI
----END SIGNATURE----
\end{lstlisting}

Im Header des MSGCRYPTED-Teils ist wie erwartet die ECB-Option gesetzt.

\subsubsection{Entschlüsselung ECB}
Die beiden Nachrichten können mit Hilfe des privaten Schlüssels des Empfängers und des öffentlichen Schlüssels des Senders entschlüsselt werden (dabei werden die in unserem Fall noch vorhandenen Original-Plaintext-Dateien überschrieben). Die SHA256-Summen der (neuen) Plaintext-Dateien entsprechenen denjenigen vor der Ver- und Entschlüsselung.

\begin{lstlisting}
[iso@iso-t530arch tmp]$ ./hsdecrypt receiverRsaPrivKey senderRsaPubKey 1k.msgEncrypted 
[iso@iso-t530arch tmp]$ ./hsdecrypt receiverRsaPrivKey senderRsaPubKey 100k.msgEncrypted 
[iso@iso-t530arch tmp]$ sha256sum 1k.msg 100k.msg
fd6df86538db0013a9f943b2d8a03d52a5d6a40cbe3243408167dc15e29a855d  1k.msg
1b13213582917f5b36751ab66bf22ae19233e259de319181acbeb64d712a3559  100k.msg
\end{lstlisting}

\subsubsection{Verschlüsselung CBC}
Die Nachrichten werden mit den selben Keys diesmal im CBC-Modus verschlüsselt und der Header des MSGCRYPTED-Teils überprüft:

\begin{lstlisting}
[iso@iso-t530arch tmp]$ ./hsencrypt RSA SHA256 AES256 CBC senderRsaPrivKey receiverRsaPubKey 1k.msg 
[iso@iso-t530arch tmp]$ ./hsencrypt RSA SHA256 AES256 CBC senderRsaPrivKey receiverRsaPubKey 100k.msg
[iso@iso-t530arch tmp]$ head -5 1k.msgEncrypted 
----BEGIN KEYCRYPTED RSA----
DiUzRSdJA72E,EdXLBpnsii33,HL4nUeyNrdUn,BL/Saw/ZQls8,IDrIpuKmMjkJ,
Hrx7GSmFoNQV,NQOf4xGv0XMJ,KtodRCoiyeZp
----END KEYCRYPTED----

----BEGIN MSGCRYPTED AES256 CBC----
....
\end{lstlisting}

\subsubsection{Entschlüsselung CBC}
Wiederum Entschlüsseln wir die Nachrichten mit den geeigneten Keys und kontrollieren die Hashwerte:

\begin{lstlisting}
[iso@iso-t530arch tmp]$ ./hsdecrypt receiverRsaPrivKey senderRsaPubKey 1k.msgEncrypted
[iso@iso-t530arch tmp]$ ./hsdecrypt receiverRsaPrivKey senderRsaPubKey 100k.msgEncrypted 
[iso@iso-t530arch tmp]$ sha256sum 1k.msg 100k.msg
fd6df86538db0013a9f943b2d8a03d52a5d6a40cbe3243408167dc15e29a855d  1k.msg
b99126267061a7ca4a63a0a59ec6e4b331d63e1dd852e2f4ba4fb72b98048a5d  100k.msg
\end{lstlisting}

Die Hashwerte stimmen überein, woraus geschlossen werden kann, dass die Ver- und Entschlüsselung der Nachrichten keinen Informationsverlust zur Folge hat.

\subsubsection{Versuch der Entschlüsselung von modifizierten Crypt-Files}
In diesem Test wird in der CBC-verschlüsselten Datei 1k.msgEncrypted eine Anpassung innerhalb einer der drei Msg-Teilen vorgenommen und versucht, die Nachricht zu entschlüsseln:

\begin{lstlisting}
[iso@iso-t530arch tmp]$ nano 1k.msgEncrypted 
[iso@iso-t530arch tmp]$ ./hsdecrypt receiverRsaPrivKey senderRsaPubKey 1k.msgEncrypted 
signature or key was wrong, exiting...
\end{lstlisting}

Wir erhalten die erwartete Fehlermeldung und die Datei wurde nicht entschlüsselt.
